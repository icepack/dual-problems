\documentclass{article}

\usepackage{amsmath}
%\usepackage{amsfonts}
\usepackage{amsthm}
%\usepackage{amssymb}
%\usepackage{mathrsfs}
%\usepackage{fullpage}
%\usepackage{mathptmx}
%\usepackage[varg]{txfonts}
\usepackage{color}
\usepackage[charter]{mathdesign}
\usepackage[pdftex]{graphicx}
%\usepackage{float}
%\usepackage{hyperref}
%\usepackage[modulo, displaymath, mathlines]{lineno}
%\usepackage{setspace}
%\usepackage[titletoc,toc,title]{appendix}
\usepackage{natbib}

%\linenumbers
%\doublespacing

\theoremstyle{definition}
\newtheorem*{defn}{Definition}
\newtheorem*{exm}{Example}

\theoremstyle{plain}
\newtheorem*{thm}{Theorem}
\newtheorem*{lem}{Lemma}
\newtheorem*{prop}{Proposition}
\newtheorem*{cor}{Corollary}

\newcommand{\argmin}{\text{argmin}}
\newcommand{\ud}{\hspace{2pt}\mathrm{d}}
\newcommand{\bs}{\boldsymbol}
\newcommand{\PP}{\mathsf{P}}

\title{Dual action principles for ice sheet dynamics}
\author{Daniel Shapero, Gonzalo Gonzalez de Diego}
\date{}

\begin{document}

\maketitle

\section{Introduction}

On space and time scales greater than 100m and 1 day, glaciers flow like a viscous, incompressible fluid with a power-law rheology \citep{greve2009dynamics}.
Ice flow is slow enough that the inertial terms in the Navier-Stokes equations can be neglected, i.e. the flow occurs at very low Reynolds number.
There are multiple equivalent ways of expressing the momentum balance equations: a conservation law, a variational form, a partial differential equation.
Each of these forms is best suited to a different type of numerical method.
The momentum balance equation for low-Reynolds number viscous fluid flow can also be derived as the optimality conditions for the velocity to be the critical point of a certain \emph{action functional} \citep{dukowicz2010consistent}.
The action functional has units of energy per unit time and can be interpreted as the rate of dissipation of thermodynamic free energy \citep{edelen1972nonlinear}.
Moreover, for many problems, including low-Reynolds number flow and heat conduction, the action is a convex functional of the unknown field.

The existence of an action principle is a special property of a very restricted class of differential equations.
Action principles are not just of theoretical interest -- we can use them to design faster, more robust numerical solvers.
First, a convex action principle implies that the second derivative is symmetric and positive-definite.
These properties guarantee convergence for Newton-type algorithms.
They also mean that we can use specialized methods, such as Cholesky factorization or the conjugate gradient method, for solving the resulting linear systems of equations \citep{nocedal2006numerical}.
These methods are not applicable to more general classes of linear systems.
Second, the action principle offers a way to measure how well an approximate solution matches the true solution and it is distinct from, say, the square norm of the residual.
The theory of convex optimization then provides us with a way to measure how close we are to convergence without having to know what the true solution is.
In \citet{shapero2021icepack}, we showed how to use this theory to design physics-based convergence criteria.

This work follows in the footsteps of \citet{dukowicz2010consistent} in studying action principles for glacier flow.
Our main contribution is the derivation of an alternative \emph{dual} action principle, distinct from that presented in \citet{dukowicz2010consistent}, from which the momentum conservation equations can be derived.
This dual form explicitly includes the stress as an unknown.
The most important feature, however, is that \textbf{the dual action principle has favorable numerical properties for shear-thinning flows} such as glacier dynamics.
To motivate the derivation, we will first look at the primal and dual forms of a linear problem arising in groundwater flow.
Then we will derive the dual form of the shallow stream equations of glacier dynamics.
Finally, we will illustrate with a numerical implementation.

In the following, we will assume familiarity with (1) the partial differential equations describing glacier flow, (2) variational calculus and the derivation of the Euler-Lagrange equations of a generic functional, and (3) convex analysis and convex duality theory.
For background reading, we refer the reader to \citet{greve2009dynamics} for glacier dynamics, \citet{weinstock1974calculus} for variational calculus, and \citet{boyd2004convex} for convex optimization.


\section{Dual action principles}

\subsection{Groundwater flow}

To illustrate our approach, we will begin by deriving a dual action principle for a simpler problem: groundwater flow in a confined aquifer.
The unknowns in groundwater flow are the fluid velocity $u$ and the pressure head $\phi$.
First, the total mass of water is conserved:
\begin{equation}
    \nabla\cdot u = f.
    \label{eq:groundwater-conservation-law}
\end{equation}
where $f$ consists of all sources and sinks.
Next, we need a constitutive law relating the velocity and hydraulic head.
In this case, the rule is Darcy's law
\begin{equation}
    u = -k\nabla\phi
    \label{eq:darcy-law}
\end{equation}
where $k$ is the hydraulic conductivity.
(Note that the conductivity could be a scalar or a rank-2 tensor.)
Substituting Darcy's law into the conservation law eliminates $u$ from the problem, leaving us with a second-order PDE for the pressure head $\phi$.
We must also add the essential Dirichlet boundary condition
\begin{equation}
    \phi|_{\partial\Omega} = g.
    \label{eq:dirichlet-bc}
\end{equation}
We can then show after the fact that this PDE is the Euler-Lagrange equation to minimize the functional
\begin{equation}
    J(\phi) = \int_\Omega\left(\frac{1}{2}k\nabla\phi\cdot\nabla\phi - f\phi\right)\ud x.
\end{equation}
Minimizing $J$ is the \emph{primal} form of the problem.

In the description above, we eliminated the velocity $u$, but we might need this quantity for other reasons, like simulating contaminant dispersal.
We could always calculate the pressure head by solving the Poisson equation and then calculate the velocity afterwards.
What if we instead wanted to solve simultaneously for both $u$ and $\phi$?
Is there some functional $L$ of both fields such that setting the derivative of $L$ to zero yields the pair of equations \eqref{eq:groundwater-conservation-law} and \eqref{eq:darcy-law}?
This idea forms the basis of \emph{dual} or \emph{mixed} formulations of the problem.
The desired functional $L$ is
\begin{equation}
    L(u, \phi) = \int_\Omega\left(\frac{1}{2}k^{-1}u\cdot u - \phi\left(\nabla\cdot u - f\right)\right)\ud x - \int_{\partial\Omega}g\, u\cdot \nu\ud\gamma.
\end{equation}
An elementary computation shows that the Euler-Lagrange equations for a critical point of $L$ are identical to the weak form of equations \eqref{eq:groundwater-conservation-law}, \eqref{eq:darcy-law}, and the boundary condition \eqref{eq:dirichlet-bc}.
In the dual problem, the hydraulic head $\phi$ plays the role of a Lagrange multiplier to enforce the conservation law $\nabla\cdot u = f$.
Note how the essential boundary condition $\phi|_{\partial\Omega} = g$ in the primal problem became a natural boundary condition in the dual problem, i.e. we could include it directly in the Lagrangian.

For our purposes, the most important thing to observe about the dual action is that \textbf{the constitutive relation is inverted}.
Where the form of the Darcy law that we started with was $u = -k\nabla\phi$, taking the derivative of $L$ with respect to $u$ and setting it equal to zero gives
\begin{equation}
    \nabla\phi = -k^{-1}u.
\end{equation}
The two forms are mathematically equivalent, so at this juncture the distinction might not seem very significant.
Indeed, for linear problems one form is as good as the other.
For nonlinear constitutive relations, however, the consequences are more drastic.

\subsection{Shallow stream approximation}

Now we turn to ice dynamics.
In the following, we will consider the \emph{shallow stream approximation}, which assumes that (1) the glacier flow has a small ratio of thickness to horizontal length scale and (2) the flow is a plug flow, i.e. the horizontal velocity is roughly constant with depth.
The key unknowns are the ice velocity $u$ and the \emph{membrane stress tensor} $M$, a rank-2 tensor with units of stress or energy density that results from applying the above approximations to the full deviatoric stress tensor.
The conservation law for membrane stress is
\begin{equation}
    \nabla\cdot hM - \tau - \rho gh\nabla s = 0
    \label{eq:membrane-stress-conservation}
\end{equation}
where $h$ and $s$ are the ice thickness and surface elevation and $\tau$ is the basal shear stress.
To close the system of equations, we must provide a constitutive relation between the membrane stress tensor and the depth-averaged strain rate tensor
\begin{equation}
    \dot\varepsilon = \frac{1}{2}\left(\nabla u + \nabla u^\top\right).
    \label{eq:strain-rate}
\end{equation}
The Glen flow law states that this is a power-law relationship, i.e.
\begin{equation}
    M = A^{-\frac{1}{n}}\sqrt{\frac{\dot\varepsilon\cdot\dot\varepsilon + \text{tr}(\dot\varepsilon)^2}{2}}^{\frac{1}{n} - 1}\left(\dot\varepsilon + \text{tr}(\dot\varepsilon)I\right)
    \label{eq:constitutive-relation}
\end{equation}
where $A$ is the fluidity coefficient and $n \approx 3$ is the Glen flow law exponent.
Finally, we need to provide some kind of sliding relation.
We will assume a generalized power law with some exponent $m$, i.e.
\begin{equation}
    \tau = -C|u|^{\frac{1}{m} - 1}u.
    \label{eq:sliding-law}
\end{equation}
Weertman sliding has $m = n$, while perfectly plasting sliding has $m = \infty$.
Recent research suggests alternative forms that transition between Weertman-type sliding at low speeds and perfectly plasting sliding at higher speeds \citep{minchew2020toward}.
For illustrative purposes equation \eqref{eq:sliding-law} is sufficient, and we will describe how to incorporate alternatives in the discussion.
We can then combine equations \eqref{eq:membrane-stress-conservation}, \eqref{eq:strain-rate}, \eqref{eq:constitutive-relation}, and \eqref{eq:sliding-law} to arrive at a second-order, nonlinear system of partial differential equations for $u$.

The shallow stream equations can be derived as the optimality conditions to find the minimum of the following action functional:
\begin{align}
    J(u) & = \int_\Omega\left(\frac{2n}{n + 1}hA^{-\frac{1}{n}}\sqrt{\frac{\dot\varepsilon\cdot\dot\varepsilon + \text{tr}(\dot\varepsilon)^2}{2}}^{\frac{1}{n} + 1} + \frac{m}{m + 1}C|u|^{\frac{1}{m} + 1} + \rho gh\nabla s\cdot u\right)\ud x  \nonumber \\
    & \qquad + \frac{1}{2}\int_\Gamma\left(\rho_Igh^2 - \rho_Wgh_W^2\right)u\cdot\nu\ud\gamma
    \label{eq:ssa-primal-action}
\end{align}
The action has units of power and, while it is no longer quadratic like the groundwater flow problem, it is still convex.

In the previous section, we derived a dual action principle for groundwater flow as follows:
\begin{enumerate}
    \item We started with a conservation law for fluid volume and a constitutive relation between fluid velocity and hydraulic head.
    \item We derived a second-order differential equation for the hydraulic head alone.
    \item We showed how this second-order equation has an action principle.
    \item We showed how there is an equivalent, dual action principle for both the velocity and hydraulic head.
\end{enumerate}
Mathematically, the derivation of this dual action principle amounts to using the convex conjugate of the action.
So far, we have a (primal) action principle for the ice velocity.
The preceding section suggests that we can likewise derive a dual action principle for both the ice velocity and membrane stress.

The dual form of the shallow stream equations adds the membrane stress tensor $M$ and the basal shear stress $\tau$ explicitly as unknowns in the problem:
\begin{align}
    L(M, \tau, u) = & \int_\Omega\Bigg\{\frac{2}{n + 1}hA\sqrt{\frac{M\cdot M - \frac{1}{d + 1}\text{tr}(M)^2}{2}}^{n + 1} + \frac{1}{m + 1}K|\tau|^{m + 1} \nonumber\\
    & \qquad\qquad + u\cdot\left(\nabla\cdot hM - \tau - \rho_Igh\nabla s\right)\Bigg\}\ud x
\end{align}
The basal shear stress is part of the PDE itself for the shallow stream equations instead of a boundary condition, as they are for the full Stokes equations.
To invert or ``dualize'' this term in the shallow stream approximation, we needed to introduce the basal shear stress explicitly as a distinct variable from the membrane stress.

The key feature of this dual formulation is that \textbf{the nature of the nonlinearity has changed}.
In the primal action shown in equation \eqref{eq:ssa-primal-action}, the nonlinearity consists of the strain rate raised to the power $\frac{1}{n} + 1$.
Since $n > 1$, the nonlinearity in the primal form has an infinite singularity in its second derivative around any velocity field with zero strain rate.
In the dual form, however, the nonlinearity consists of the stress tensor raised to the power $n + 1$.
Around zero stress, the second derivative of the action with respect to the stress is zero instead of infinity; see figure \ref{fig:primal-vs-dual}.

\begin{figure}[t]
    \includegraphics[width=0.48\linewidth]{demos/singularity/primal.pdf}\includegraphics[width=0.48\linewidth]{demos/singularity/dual.pdf}
    \caption{The viscous part of the action for the primal problem as a function of the strain rate (left) and for the dual problem as a function of the stress (right).
    The second derivative of the viscous dissipation goes to infinity near zero strain rate for the primal problem, but to zero near zero stress for the dual problem.}
    \label{fig:primal-vs-dual}
\end{figure}


\section{Discretization}

Roughly any conforming finite element basis is stable for the primal form of the Poisson equation.
The most common choice is to use piecewise-continuous polynomials of a given degree $k$ on triangles, or the tensor product of polynomials on quads.
We will refer to this basis as $CG(k)$.
While mixed formulations have many advantages, the key challenge to overcome is that most choices of basis are unstable -- the resulting linear systems are either singular or their inverses have unbounded norm in the limit as the mesh is refined.
For example, using $CG(k)$ elements for the pressure and the product $CG(k)^2$ for the velocity is an unstable discretization of the dual form of the groundwater problem.
For a degree-1 basis, we can remedy this situation by enriching the velocity space with cubic \emph{bubble} functions $B(3)$.
Using the space $(CG(1) \oplus B(3))^2$ for the velocities and $CG(1)$ for the hydraulic head is stable for mixed Poisson \citep{boffi2013mixed}.
We could also have used the \emph{discontinuous} space $DG(k)$ for the pressures and Raviart-Thomas elements $RT(k)$ for the velocity.
The Raviart-Thomas basis functions have continuous normal components across element boundaries, and are thus a conforming discretization of the Sobolev space $H^{\text{div}}(\Omega)$.

The shallow stream equations are formally similar to the 2D elasticity equations and thus much of the theory for discretizing the mixed elasticity system applies here.
Stable discretization of the mixed elasticity problem is much more complicated than the mixed form of the Poisson equation because we have an additional invariant to enforce: the symmetry of the stress tensor.
There are three viable approaches:
\begin{enumerate}
    \item Use continuous basis functions for both the velocity and the stress tensor and enrich the stress space with bubble functions \citep{brezzi1993mixed}.
        The basis functions for the stress tensor are taken to be symmetric.
    \item Use discontinuous basis functions for velocity and the Raviart-Thomas or similar basis for the rows of the stress tensor.
        Enforce the symmetry of the stress tensor weakly by addition of another Lagrange multiplier; the stress basis is not symmetric a priori \citep{arnold1984peers}.
    \item Use the Arnold-Winther element, which is conforming for mixed elasticity but which requires many degrees of freedom \citep{arnold2002mixed}.
\end{enumerate}
In the demonstrations that follow, we used both bubble functions and the AW element.
Firedrake includes an implementation of the AW element, which is relatively uncommon among software packages for finite element analysis \citep{aznaran2021transformations}.
\textcolor{red}{Actually test it with AW...}


\section{Demonstrations}

We implemented a solver for the dual form of the SSA using the Firedrake package.
To solve for the velocity and membrane stress tensor, we proceed in two steps.
We start with a continuation-type approach -- make the problem linear by setting $n = 1$ to get a starting guess at the velocity and membrane stress.
Next, we use this guess to solve the full nonlinear problem using a Newton trust region method.

The trust region method has distinct advantages over the more commonly applied line search methods for problems where the second derivative can have zero degeneracies.
The trust region method regularizes over the zero degeneracy by adding a small multiple of some positive-definite matrix \citep{conn2000trust}.
The key point here is that the inverse of this matrix multiplies the true derivative of the Lagrangian, so despite this regularization, the approximate solutions converge to the critical point of the true, unmodified Lagrangian.
By contrast, the conventional methods for the primal problem modify the action functional in order to eliminate the infinite singularity.
The primal problem grows more ill-conditioned in the limit as this regularization is shrunk to zero.

\subsection{Linear ice shelf}

\begin{figure}[h]
    \begin{center}
        \includegraphics[width=0.75\linewidth]{demos/ice-shelf/results.eps}
    \end{center}
    \caption{Relative $L^2$-norm errors for approximate solutions to the analytical ice shelf test case.
    Convergence rate obtained through a log-log fit of errors against mesh size.}
    \label{fig:linear-ice-shelf-convergence-rate}
\end{figure}

As a first check that we implemented the dual form of the SSA correctly, we used the exact solution for a floating ice shelf with thickness
\begin{equation}
    h = h_0 - \delta h \cdot x / L_x
\end{equation}
in a domain of length $L_x$.
With a constant value of the fluidity coefficient, the velocity in the $x$ direction is
\begin{equation}
    u_x = u_0 + L_x A \left(\frac{\rho g h_0}{4}\right)^n\left(1 - \left(1 - \frac{\delta h \cdot x}{h_0\cdot L_x}\right)^{n + 1}\right).
\end{equation}
We used a 2D domain in order to make sure that the numerical solution has no variation in the $y$ direction.
The relative error in the $L^2$ norm of the numerical solutions has the expected asymptotic convergence rate of $\mathscr{O}(\delta x^2)$ for continuous piecewise linear velocity elements; see figure \ref{fig:linear-ice-shelf-convergence-rate}.

The inflow boundary conditions are natural in the dual form.
We used a nonlinear Nitsche's method to impose the essential boundary conditions at the calving terminus; see appendix \ref{sec:boundary-conditions} for more details.

\subsection{Bodvarsson solution}

We tested our implementation using the exact solution for a marine ice sheet described in \citet{bueler2014exact}.
The geometry in \citet{bueler2014exact} extends from $x = 0$ to $x = L$ with the understanding that the solution is symmetric.
We instead used a domain extending from $x = -L$ to $x = +L$ to include the dangerous point at $x = 0$ in the interior.

\subsection{Synthetic ice shelf}

As a more realistic test case, we used a synthetic ice shelf flowing over an ice rise.
We made the elevation of the ice rise above sea level and its basal friction high enough to cause the ice to stagnate on top of the rise.
Near the ice rise, both the basal sliding velocity and the strain rate approach zero, so the solver must be able to cope with the resulting degeneracy.


\section{Discussion}

\textcolor{red}{Finish this...}

Using the dual formulation of the problem has several advantages.
The stress tensor may be an input to other parts of the physics.
For example, it is part of the source terms for both heat and damage.
The dual formulation includes the stress explicitly as an unknown in the problem.
Solving for it directly offers greater accuracy than computing it after the fact from the velocity field.
Second, the dual formulation reverses the behavior of all the nonlinearities around the zero-disturbance state.
Infinite singularities in the primal formulation, which can only be dealt with by fudging the problem itself, become zero degeneracies in the dual.
These degeneracies are still a challenge.
But the problem, with no modifications, is amenable to solution by continuation methods.
We believe that trust region methods might work as well and this remains to be explored.

The dual formulation does come with several disadvantages.
The number of unknowns in the dual formulation is much greater than in the primal form, thus putting more pressure on computer memory.
The resulting linear systems are indefinite rather than positive-definite.
Finally, the choice of finite element basis is much more delicate.
The dual formulation offers alternative possibilities for incorporating plastic yield as an inequality constraint.
\textcolor{red}{Elaborate...}


\appendix

\section{Boundary conditions} \label{sec:boundary-conditions}

The boundary condition at the terminus of a glacier is a Neumann-type condition on the membrane stress.
Physically, an ice cliff experiences a pressure that pushes the terminus out, while any ocean water or ice melange imposes a back-pressure.
The boundary condition is
\begin{equation}
    M\cdot\nu = p_t\nu
\end{equation}
where $p_t$ is the terminal pressure.
For a glacier calving into a body of water that rises up to a relative height $f$ through the ice column,
\begin{equation}
    p_t = \frac{1}{2}\left(\rho_I - \rho_Wf^2\right)gh.
\end{equation}
A land-terminating glacier has $f = 0$, while for a floating ice shelf, $f = \rho_I / \rho_W$.

The dual formulation not only changes the constitutive relations but also changes which boundary conditions are natural or essential.
For the primal problem, boundary conditions on the normal stress at the calving terminus or along the side walls are natural, while the inflow boundary conditions are essential.
These are reversed in the dual problem, but enforcing the essential boundary condition only on the normal component of the stress at the terminus by fixing the values of some element degrees of freedom is much more difficult than setting all components of the inflow ice velocity.
Treating this boundary condition as a constraint, we could enforce it through a pure penalty method, which would destroy the condition number of the system, or we could try a Lagrange multiplier, but choosing finite element spaces for the multiplier is infeasible.
Finally, we could try an augmented Lagrangian-type approach which combines the penalty and Lagrange multiplier methods.
We could then note that the Lagrange multiplier is, in this case, exactly equal to the ice velocity.
This final step is the idea behind \emph{Nitsche's method}.

Let $\Gamma_t$ denote the calving terminus part of the domain boundary.
For the experiments in this paper, we used the following modified action functional to enforce the boundary condition on the membrane stress at the ice terminus:
\begin{align}
    L_\alpha(M, \tau, u) & = L(M, \tau, u) - \int_{\Gamma_t}u\cdot h\left(M\cdot\nu - p_t\nu\right)\ud\gamma \nonumber\\
    & \qquad + \int_{\Gamma_t}\frac{2\alpha\ell}{n + 1}hA\left|M\cdot\nu - p_t\nu\right|^{n + 1}\ud\gamma
    \label{eq:nitsche-functional}
\end{align}
where $\alpha$ is a penalty parameter that we need to pick and $\ell$ is the local cell size.

For linear problems, the right value of $\alpha$ can be calculated ahead of time using sharp estimates for the constant in finite element inverse inequalities \citep{warburton2003constants}.
The situation is less clear for problems with power-law nonlinearities such as the Glen flow law.
To our knowledge, there is no published convexity proof for a Nitsche-type method applied to problems with general power-law nonlinearities.
We \emph{conjecture} that equation \eqref{eq:nitsche-functional} is the correct form of the action functional and that $\alpha$ can be calculated by using the Riesz-Thorin theorem applied to the original $L^2$-inverse inequalities.
We found experimentally that all problems using the modified functional in equation \eqref{eq:nitsche-functional} were solvable.
Nonetheless, we do not have a formal proof and this will be the subject of future work.


\pagebreak

\bibliographystyle{plainnat}
\bibliography{dual-problems.bib}

\end{document}
